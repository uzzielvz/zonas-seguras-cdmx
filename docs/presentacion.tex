\documentclass[aspectratio=169]{beamer}
\usetheme{Madrid}
\usecolortheme{default}

\usepackage[utf8]{inputenc}
\usepackage[spanish]{babel}
\usepackage{graphicx}
\usepackage{listings}
\usepackage{xcolor}

% Configuración de código
\lstset{
  basicstyle=\ttfamily\small,
  keywordstyle=\color{blue},
  stringstyle=\color{red},
  commentstyle=\color{green},
  breaklines=true,
  frame=single
}

% Información del documento
\title{ZonaSegura CDMX}
\subtitle{Mapa Colaborativo de Seguridad Ciudadana}
\author{Proyecto de Sistemas de Información Geográfica}
\date{\today}

\begin{document}

% Portada
\frame{\titlepage}

% Tabla de contenidos
\begin{frame}{Contenido}
  \tableofcontents
\end{frame}

% ============================================
% SECCIÓN 1: INTRODUCCIÓN
% ============================================
\section{Introducción}

\begin{frame}{¿Qué es ZonaSegura CDMX?}
  \begin{itemize}
    \item \textbf{Aplicación web interactiva} que combina:
    \begin{itemize}
      \item Datos oficiales de delitos de la Fiscalía General de Justicia CDMX
      \item Reportes en tiempo real de ciudadanos
    \end{itemize}
    \item \textbf{Objetivo}: Visualizar de manera clara y rápida:
    \begin{itemize}
      \item Dónde están ocurriendo los delitos
      \item Zonas de mayor riesgo
      \item Cambios en la seguridad por colonia
    \end{itemize}
    \item \textbf{Enfoque}: Robos, asaltos y homicidios (para visitantes)
  \end{itemize}
\end{frame}

\begin{frame}{Datos del Proyecto}
  \begin{columns}
    \column{0.5\textwidth}
    \begin{block}{Datos Oficiales}
      \begin{itemize}
        \item \textbf{808,871} registros totales
        \item \textbf{777,535} con coordenadas válidas (96.1\%)
        \item \textbf{55,561} delitos filtrados (robos/asaltos/homicidios desde 2019)
        \item Rango: 2019 - 2019
      \end{itemize}
    \end{block}
    
    \column{0.5\textwidth}
    \begin{block}{Desglose por Tipo}
      \begin{itemize}
        \item \textbf{15,761} Robos
        \item \textbf{39,060} Asaltos
        \item \textbf{740} Homicidios
      \end{itemize}
    \end{block}
  \end{columns}
  
  \vspace{0.5cm}
  \begin{alertblock}{Fuente}
    Fiscalía General de Justicia de la Ciudad de México (FGJ-CDMX)
  \end{alertblock}
\end{frame}

% ============================================
% SECCIÓN 2: ARQUITECTURA
% ============================================
\section{Arquitectura del Sistema}

\begin{frame}{Arquitectura General}
  \begin{center}
    \includegraphics[width=0.8\textwidth]{../diagrams/arquitectura.png}
  \end{center}
  
  \begin{description}
    \item[Frontend] React + TypeScript + Vite
    \item[Visualización] Leaflet.js + react-leaflet
    \item[Análisis Geoespacial] Turf.js
    \item[Almacenamiento] LocalStorage (demo)
  \end{description}
\end{frame}

\begin{frame}{Stack Tecnológico}
  \begin{columns}
    \column{0.5\textwidth}
    \begin{block}{Frontend}
      \begin{itemize}
        \item \textbf{React 18.2} - Framework UI
        \item \textbf{TypeScript 5.2} - Tipado estático
        \item \textbf{Vite 5.0} - Build tool
        \item \textbf{Tailwind CSS 3.3} - Estilos
      \end{itemize}
    \end{block}
    
    \column{0.5\textwidth}
    \begin{block}{Librerías Geoespaciales}
      \begin{itemize}
        \item \textbf{Leaflet 1.9} - Mapas
        \item \textbf{react-leaflet 4.2} - Integración React
        \item \textbf{leaflet.heat 0.2} - Mapas de calor
        \item \textbf{Turf.js 6.5} - Análisis geoespacial
      \end{itemize}
    \end{block}
  \end{columns}
\end{frame}

\begin{frame}{Estructura de Componentes}
  \begin{block}{Organización Modular}
    \begin{itemize}
      \item \texttt{src/components/Map/} - Componentes del mapa
      \begin{itemize}
        \item MapView.tsx - Contenedor principal
        \item HeatmapLayer.tsx - Mapa de calor
        \item SafetyZonesLayer.tsx - Zonas de seguridad
        \item ReportMarkers.tsx - Marcadores de reportes
      \end{itemize}
      \item \texttt{src/components/ReportForm/} - Formulario de reportes
      \item \texttt{src/components/Sidebar/} - Panel de filtros
      \item \texttt{src/hooks/} - Custom hooks (useDelitosData)
      \item \texttt{src/utils/} - Utilidades (reportes.ts)
    \end{itemize}
  \end{block}
\end{frame}

% ============================================
% SECCIÓN 3: PROCESO DE DESARROLLO
% ============================================
\section{Proceso de Desarrollo}

\begin{frame}{Fases del Desarrollo}
  \begin{enumerate}
    \item \textbf{Análisis de Datos}
    \begin{itemize}
      \item Procesamiento del CSV (808k+ registros)
      \item Validación de coordenadas
      \item Clasificación de delitos
    \end{itemize}
    
    \item \textbf{Configuración del Proyecto}
    \begin{itemize}
      \item Setup React + TypeScript + Vite
      \item Integración de Leaflet
      \item Configuración de Tailwind CSS
    \end{itemize}
    
    \item \textbf{Visualización Base}
    \begin{itemize}
      \item Mapa de CDMX con límites exactos
      \item Sistema de coordenadas
    \end{itemize}
  \end{enumerate}
\end{frame}

\begin{frame}{Fases del Desarrollo (Cont.)}
  \begin{enumerate}
    \setcounter{enumi}{3}
    \item \textbf{Sistema de Reportes}
    \begin{itemize}
      \item Formulario de captura
      \item Almacenamiento en LocalStorage
      \item Marcadores personalizados por tipo
    \end{itemize}
    
    \item \textbf{Mapa de Calor}
    \begin{itemize}
      \item Conversión CSV → GeoJSON
      \item Implementación con leaflet.heat
      \item Controles de intensidad
    \end{itemize}
    
    \item \textbf{Zonas de Seguridad}
    \begin{itemize}
      \item Algoritmo de clasificación
      \item Grid de celdas (1km²)
      \item Sistema de colores (verde/rojo)
    \end{itemize}
  \end{enumerate}
\end{frame}

% ============================================
% SECCIÓN 4: FUNCIONALIDADES
% ============================================
\section{Funcionalidades}

\begin{frame}{Características Principales}
  \begin{columns}
    \column{0.5\textwidth}
    \begin{block}{Visualización}
      \begin{itemize}
        \item Mapa de calor interactivo
        \item Zonas de seguridad (verde/rojo)
        \item Marcadores de reportes ciudadanos
        \item Límites exactos de CDMX
      \end{itemize}
    \end{block}
    
    \column{0.5\textwidth}
    \begin{block}{Reportes Ciudadanos}
      \begin{itemize}
        \item Captura rápida (< 10 seg)
        \item Geolocalización o clic en mapa
        \item Tipos: robo, asalto, homicidio, acoso, otro
        \item Foto opcional
      \end{itemize}
    \end{block}
  \end{columns}
  
  \vspace{0.5cm}
  \begin{block}{Filtros Avanzados}
    \begin{itemize}
      \item Por tipo de delito
      \item Por fecha (últimos 30 días)
      \item Por alcaldía
      \item Toggle de capas (calor, buffers, reportes)
    \end{itemize}
  \end{block}
\end{frame}

\begin{frame}{Algoritmo de Zonas de Seguridad}
  \begin{block}{Proceso de Clasificación}
    \begin{enumerate}
      \item \textbf{División en Grid}: CDMX dividida en celdas de 1km²
      \item \textbf{Conteo de Delitos}: Contar delitos por celda
      \item \textbf{Cálculo de Densidad}: Delitos/km²
      \item \textbf{Clasificación}:
      \begin{itemize}
        \item Verde: 0 delitos (muy seguro)
        \item Verde claro: < 5 delitos/km² (seguro)
        \item Amarillo: 5-15 delitos/km² (moderado)
        \item Naranja: 15-30 delitos/km² (peligroso)
        \item Rojo: > 30 delitos/km² (muy peligroso)
      \end{itemize}
    \end{enumerate}
  \end{block}
\end{frame}

\begin{frame}{Mapa de Calor}
  \begin{columns}
    \column{0.5\textwidth}
    \begin{block}{Características}
      \begin{itemize}
        \item 55,561 puntos de datos
        \item Gradiente de colores:
        \begin{itemize}
          \item Azul → Baja intensidad
          \item Amarillo → Media intensidad
          \item Rojo → Alta intensidad
        \end{itemize}
        \item Controles ajustables:
        \begin{itemize}
          \item Intensidad (10-100\%)
          \item Solo zonas críticas
          \item Optimización de rendimiento
        \end{itemize}
      \end{itemize}
    \end{block}
    
    \column{0.5\textwidth}
    \begin{block}{Optimizaciones}
      \begin{itemize}
        \item Muestreo automático (>10k puntos)
        \item Filtrado por tipo de delito
        \item Filtrado por fecha
        \item Renderizado dinámico
      \end{itemize}
    \end{block}
  \end{columns}
\end{frame}

% ============================================
% SECCIÓN 5: HERRAMIENTAS Y TECNOLOGÍAS
% ============================================
\section{Herramientas y Tecnologías}

\begin{frame}{Herramientas de Desarrollo}
  \begin{block}{Entorno de Desarrollo}
    \begin{itemize}
      \item \textbf{VS Code} - Editor de código
      \item \textbf{Node.js} - Runtime de JavaScript
      \item \textbf{npm} - Gestor de paquetes
      \item \textbf{Git} - Control de versiones
    \end{itemize}
  \end{block}
  
  \begin{block}{Build y Deploy}
    \begin{itemize}
      \item \textbf{Vite} - Build tool rápido
      \item \textbf{TypeScript} - Compilador
      \item \textbf{ESLint} - Linter
      \item \textbf{PostCSS + Autoprefixer} - Procesamiento CSS
    \end{itemize}
  \end{block}
\end{frame}

\begin{frame}{Procesamiento de Datos}
  \begin{block}{Scripts Python}
    \begin{itemize}
      \item \texttt{csv\_to\_geojson.py} - Conversión de datos
      \begin{itemize}
        \item Filtrado por tipo (robos/asaltos/homicidios)
        \item Validación de coordenadas
        \item Clasificación de delitos graves
        \item Generación de GeoJSON
      \end{itemize}
      \item Scripts de análisis:
      \begin{itemize}
        \item Análisis de estructura del CSV
        \item Estadísticas por tipo de delito
        \item Validación de coordenadas
      \end{itemize}
    \end{itemize}
  \end{block}
  
  \begin{alertblock}{Resultado}
    CSV (808k registros, >200MB) → GeoJSON (55k delitos, 30MB)
  \end{alertblock}
\end{frame}

\begin{frame}{Librerías Geoespaciales}
  \begin{columns}
    \column{0.5\textwidth}
    \begin{block}{Leaflet.js}
      \begin{itemize}
        \item Biblioteca de mapas open-source
        \item Ligera y rápida
        \item Compatible con móviles
        \item Plugins extensibles
      \end{itemize}
    \end{block}
    
    \column{0.5\textwidth}
    \begin{block}{Turf.js}
      \begin{itemize}
        \item Análisis geoespacial en JavaScript
        \item Operaciones: buffers, unions, convex hull
        \item Cálculos de distancia
        \item Validación de geometrías
      \end{itemize}
    \end{block}
  \end{columns}
  
  \vspace{0.5cm}
  \begin{block}{leaflet.heat}
    Plugin para visualización de mapas de calor basado en densidad de puntos
  \end{block}
\end{frame}

% ============================================
% SECCIÓN 6: UTILIDAD Y FUNCIONAMIENTO
% ============================================
\section{Utilidad y Funcionamiento}

\begin{frame}{Casos de Uso}
  \begin{block}{Para Visitantes}
    \begin{itemize}
      \item Identificar zonas seguras antes de visitar
      \item Evitar áreas de alto riesgo
      \item Planificar rutas seguras
      \item Consultar seguridad en tiempo real
    \end{itemize}
  \end{block}
  
  \begin{block}{Para Ciudadanos}
    \begin{itemize}
      \item Reportar incidentes rápidamente
      \item Contribuir a la seguridad colectiva
      \item Monitorear cambios en su colonia
      \item Comparar datos oficiales vs. reportes ciudadanos
    \end{itemize}
  \end{block}
  
  \begin{block}{Para Autoridades}
    \begin{itemize}
      \item Visualizar patrones de delincuencia
      \item Identificar zonas críticas
      \item Validar reportes ciudadanos
      \item Análisis de tendencias
    \end{itemize}
  \end{block}
\end{frame}

\begin{frame}{Flujo de Datos}
  \begin{block}{Datos Oficiales}
    \begin{enumerate}
      \item CSV de la Fiscalía (808k registros)
      \item Procesamiento con Python
      \item Filtrado y validación
      \item Conversión a GeoJSON (55k delitos)
      \item Carga en el frontend
      \item Visualización en el mapa
    \end{enumerate}
  \end{block}
  
  \begin{block}{Reportes Ciudadanos}
    \begin{enumerate}
      \item Usuario completa formulario
      \item Captura de ubicación (GPS o clic)
      \item Almacenamiento en LocalStorage
      \item Renderizado inmediato en mapa
      \item Marcador con icono según tipo
    \end{enumerate}
  \end{block}
\end{frame}

\begin{frame}{Sistema de Filtros}
  \begin{columns}
    \column{0.5\textwidth}
    \begin{block}{Filtros de Datos}
      \begin{itemize}
        \item \textbf{Tipo de delito}:
        \begin{itemize}
          \item Homicidios
          \item Asaltos
          \item Robos
          \item Acoso
          \item Otro
        \end{itemize}
        \item \textbf{Fecha}: Últimos 30 días (por defecto)
        \item \textbf{Alcaldía}: Filtro por municipio
      \end{itemize}
    \end{block}
    
    \column{0.5\textwidth}
    \begin{block}{Filtros de Visualización}
      \begin{itemize}
        \item \textbf{Mapa de Calor}: On/Off
        \begin{itemize}
          \item Intensidad (10-100\%)
          \item Solo zonas críticas
          \item Optimización
        \end{itemize}
        \item \textbf{Zonas de Seguridad}: On/Off
        \item \textbf{Reportes Ciudadanos}: On/Off
        \item \textbf{Buffers de Riesgo}: On/Off (futuro)
      \end{itemize}
    \end{block}
  \end{columns}
\end{frame}

% ============================================
% SECCIÓN 7: ASPECTOS TÉCNICOS
% ============================================
\section{Aspectos Técnicos}

\begin{frame}{Optimizaciones Implementadas}
  \begin{block}{Rendimiento}
    \begin{itemize}
      \item \textbf{Muestreo}: Reducción de puntos para mapas de calor (>10k)
      \item \textbf{Lazy Loading}: Carga de datos bajo demanda
      \item \textbf{Memoria}: Limpieza de capas al desactivar
      \item \textbf{Renderizado}: Actualización solo cuando es necesario
    \end{itemize}
  \end{block}
  
  \begin{block}{Experiencia de Usuario}
    \begin{itemize}
      \item \textbf{Feedback Visual}: Animaciones y transiciones
      \item \textbf{Responsive}: Adaptable a móviles
      \item \textbf{Accesibilidad}: Tooltips informativos
      \item \textbf{Validación}: Formularios con validación en tiempo real
    \end{itemize}
  \end{block}
\end{frame}

\begin{frame}{Manejo de Datos Geoespaciales}
  \begin{block}{Validación de Coordenadas}
    \begin{itemize}
      \item Rango de CDMX: Lat 19.0-19.7, Lng -99.4 a -98.9
      \item Eliminación de coordenadas inválidas (0,0)
      \item Validación de formato GeoJSON
    \end{itemize}
  \end{block}
  
  \begin{block}{Operaciones Geoespaciales}
    \begin{itemize}
      \item \textbf{Unión de polígonos}: Límites de CDMX
      \item \textbf{Grid de celdas}: Zonas de seguridad
      \item \textbf{Cálculo de densidad}: Delitos por km²
      \item \textbf{Buffers}: Futuro para zonas de riesgo
    \end{itemize}
  \end{block}
\end{frame}

\begin{frame}[fragile]{Ejemplo de Código: Hook de Datos}
\begin{lstlisting}[language=JavaScript]
export function useDelitosData() {
  const [delitos, setDelitos] = useState<Delito[]>([])
  const [loading, setLoading] = useState(true)

  useEffect(() => {
    const loadDelitos = async () => {
      const response = await fetch('/data/delitos-cdmx.geojson')
      const data = await response.json()
      
      const delitosConvertidos = data.features.map((feature) => ({
        id: `delito-${index}`,
        tipo: props.tipo as TipoDelito,
        coordenadas: [coords[1], coords[0]],
        // ... más propiedades
      }))
      
      setDelitos(delitosConvertidos)
    }
    loadDelitos()
  }, [])

  return { delitos, loading, filtrarDelitos, getHeatmapData }
}
\end{lstlisting}
\end{frame}

% ============================================
% SECCIÓN 8: CONCLUSIONES Y FUTURO
% ============================================
\section{Conclusiones}

\begin{frame}{Logros del Proyecto}
  \begin{block}{Técnicos}
    \begin{itemize}
      \item Procesamiento exitoso de 808k+ registros
      \item Visualización eficiente de 55k+ puntos
      \item Sistema modular y escalable
      \item Integración de múltiples tecnologías geoespaciales
    \end{itemize}
  \end{block}
  
  \begin{block}{Funcionales}
    \begin{itemize}
      \item Mapa de calor interactivo
      \item Sistema de clasificación de zonas
      \item Reportes ciudadanos funcionales
      \item Filtros avanzados operativos
    \end{itemize}
  \end{block}
\end{frame}

\begin{frame}{Mejoras Futuras}
  \begin{block}{Corto Plazo}
    \begin{itemize}
      \item Implementar buffers de riesgo (150m)
      \item Backend para reportes (actualmente LocalStorage)
      \item Sistema de verificación de reportes
      \item Exportación de datos
    \end{itemize}
  \end{block}
  
  \begin{block}{Largo Plazo}
    \begin{itemize}
      \item API para datos en tiempo real
      \item Machine Learning para predicción
      \item App móvil nativa
      \item Integración con otras fuentes de datos
      \item Sistema de alertas por proximidad
    \end{itemize}
  \end{block}
\end{frame}

\begin{frame}{Impacto Social}
  \begin{block}{Beneficios}
    \begin{itemize}
      \item \textbf{Transparencia}: Datos oficiales accesibles
      \item \textbf{Empoderamiento}: Ciudadanos informados
      \item \textbf{Colaboración}: Reportes ciudadanos
      \item \textbf{Prevención}: Identificación de zonas de riesgo
    \end{itemize}
  \end{block}
  
  \begin{alertblock}{Objetivo Final}
    Crear una herramienta que ayude a ciudadanos y visitantes a tomar decisiones informadas sobre seguridad en la Ciudad de México
  \end{alertblock}
\end{frame}

% ============================================
% FINAL
% ============================================
\begin{frame}{Preguntas}
  \begin{center}
    \Huge ¿Preguntas?
    
    \vspace{1cm}
    
    \Large Gracias por su atención
  \end{center}
\end{frame}

\end{document}

