\documentclass[aspectratio=169]{beamer}

% Tema minimalista - fondo blanco, texto negro
\usetheme{default}
\usecolortheme{default}

% Configuración de colores minimalista
\setbeamercolor{background canvas}{bg=white}
\setbeamercolor{normal text}{fg=black}
\setbeamercolor{title}{fg=black}
\setbeamercolor{frametitle}{fg=black}
\setbeamercolor{structure}{fg=black}
\setbeamercolor{item}{fg=black}
\setbeamercolor{block title}{fg=black,bg=white}
\setbeamercolor{block body}{fg=black,bg=white}
\setbeamercolor{block title alerted}{fg=black,bg=white}
\setbeamercolor{block body alerted}{fg=black,bg=white}

% Eliminar elementos decorativos
\setbeamertemplate{navigation symbols}{}
\setbeamertemplate{headline}{}
\setbeamertemplate{footline}{}

% Sin sombras ni decoraciones
\setbeamertemplate{blocks}[default]
\setbeamertemplate{sidebar left}{}
\setbeamertemplate{sidebar right}{}

\usepackage[utf8]{inputenc}
\usepackage[spanish]{babel}
\usepackage{graphicx}
\usepackage{listings}
\usepackage{xcolor}

% Configuración de código - minimalista
\lstset{
  basicstyle=\ttfamily\footnotesize,
  keywordstyle=\color{black}\bfseries,
  stringstyle=\color{black},
  commentstyle=\color{gray},
  breaklines=true,
  frame=none,
  backgroundcolor=\color{white}
}

% Información del documento
\title{ZonaSegura CDMX}
\subtitle{Mapa Colaborativo de Seguridad Ciudadana}
\author{Proyecto de Sistemas de Información Geográfica}
\date{\today}

\begin{document}

% Portada
\begin{frame}
  \titlepage
\end{frame}

% Tabla de contenidos
\begin{frame}{Contenido}
  \tableofcontents
\end{frame}

% ============================================
% SECCIÓN 1: INTRODUCCIÓN
% ============================================
\section{Introducción}

\begin{frame}{Proyecto ZonaSegura CDMX}
  \begin{itemize}
    \item Aplicación web para visualizar datos de seguridad en la Ciudad de México
    \item Combina datos oficiales de la Fiscalía General de Justicia con reportes ciudadanos
    \item Objetivo: ayudar a visitantes y ciudadanos a identificar zonas de riesgo
  \end{itemize}
  
  \vspace{0.5cm}
  
  \textbf{Datos procesados:}
  \begin{itemize}
    \item 808,871 registros originales
    \item 55,561 delitos filtrados (robos, asaltos, homicidios desde 2019)
  \end{itemize}
\end{frame}

% ============================================
% SECCIÓN 2: PROCESO DE DESARROLLO
% ============================================
\section{Proceso de Desarrollo}

\begin{frame}{Fase 1: Análisis y Preparación de Datos}
  \begin{enumerate}
    \item \textbf{Obtención de datos}
    \begin{itemize}
      \item CSV con 808,871 registros de delitos (Fiscalía CDMX)
      \item Validación de estructura y formatos
    \end{itemize}
    
    \item \textbf{Procesamiento inicial}
    \begin{itemize}
      \item Script Python para análisis exploratorio
      \item Identificación de tipos de delitos relevantes
      \item Verificación de campos de coordenadas
    \end{itemize}
  \end{enumerate}
\end{frame}

\begin{frame}{Fase 2: Transformación de Datos}
  \begin{enumerate}
    \item \textbf{Validación de coordenadas}
    \begin{itemize}
      \item Filtrado de coordenadas inválidas (0,0 o fuera de CDMX)
      \item Verificación de rango geográfico: Lat 19.0-19.7, Lng -99.4 a -98.9
      \item Resultado: 777,535 registros con coordenadas válidas (96.1\%)
    \end{itemize}
    
    \item \textbf{Filtrado por tipo}
    \begin{itemize}
      \item Selección de delitos relevantes: robos, asaltos, homicidios
      \item Filtrado por fecha: delitos desde 2019 en adelante
      \item Resultado: 55,561 delitos filtrados
    \end{itemize}
  \end{enumerate}
\end{frame}

\begin{frame}{Fase 3: Conversión a GeoJSON}
  \begin{enumerate}
    \item \textbf{Script de conversión} (\texttt{csv\_to\_geojson.py})
    \begin{itemize}
      \item Lectura del CSV procesado
      \item Clasificación de delitos por tipo
      \item Validación de cada registro
      \item Generación de archivo GeoJSON
    \end{itemize}
    
    \item \textbf{Resultado}
    \begin{itemize}
      \item Reducción de tamaño: 200MB (CSV) → 30MB (GeoJSON)
      \item Formato compatible con librerías de mapas web
      \item 55,561 features con coordenadas y metadatos
    \end{itemize}
  \end{enumerate}
\end{frame}

\begin{frame}{Fase 4: Configuración del Proyecto Base}
  \begin{enumerate}
    \item \textbf{Inicialización}
    \begin{itemize}
      \item Creación del proyecto con Vite + React + TypeScript
      \item Configuración de estructura de carpetas
      \item Instalación de dependencias base
    \end{itemize}
    
    \item \textbf{Integración de librerías}
    \begin{itemize}
      \item Leaflet.js para visualización de mapas
      \item react-leaflet para componentes React
      \item Tailwind CSS para estilos
      \item Turf.js para análisis geoespacial
    \end{itemize}
  \end{enumerate}
\end{frame}

\begin{frame}{Fase 5: Implementación del Mapa Base}
  \begin{enumerate}
    \item \textbf{Visualización inicial}
    \begin{itemize}
      \item Configuración del contenedor de mapa
      \item Integración de capa base (OpenStreetMap)
      \item Centrado en coordenadas de CDMX
    \end{itemize}
    
    \item \textbf{Límites de CDMX}
    \begin{itemize}
      \item Carga del GeoJSON de límites de alcaldías
      \item Unión de polígonos usando Turf.js para obtener contorno exterior
      \item Renderizado de límite como línea roja
    \end{itemize}
  \end{enumerate}
\end{frame}

\begin{frame}{Fase 6: Sistema de Reportes Ciudadanos}
  \begin{enumerate}
    \item \textbf{Formulario de captura}
    \begin{itemize}
      \item Componente modal con campos: tipo, ubicación, descripción, foto
      \item Validación de campos requeridos
      \item Iconos personalizados por tipo de delito
    \end{itemize}
    
    \item \textbf{Selección de ubicación}
    \begin{itemize}
      \item Opción 1: Geolocalización del dispositivo
      \item Opción 2: Selección manual en el mapa (con zoom automático)
      \item Guardado de coordenadas
    \end{itemize}
    
    \item \textbf{Almacenamiento}
    \begin{itemize}
      \item LocalStorage para persistencia local
      \item Renderizado inmediato en el mapa con marcadores personalizados
    \end{itemize}
  \end{enumerate}
\end{frame}

\begin{frame}{Fase 7: Integración de Datos Oficiales}
  \begin{enumerate}
    \item \textbf{Carga de datos}
    \begin{itemize}
      \item Custom hook \texttt{useDelitosData} para gestión de estado
      \item Fetch asíncrono del GeoJSON procesado
      \item Conversión a formato interno del componente
    \end{itemize}
    
    \item \textbf{Filtrado dinámico}
    \begin{itemize}
      \item Filtros por tipo de delito (homicidios, asaltos, robos)
      \item Filtros por rango de fechas (últimos 30 días por defecto)
      \item Filtros por alcaldía
      \item Optimización con \texttt{useMemo} para evitar recálculos innecesarios
    \end{itemize}
  \end{enumerate}
\end{frame}

\begin{frame}{Fase 8: Implementación del Mapa de Calor}
  \begin{enumerate}
    \item \textbf{Plugin de heatmap}
    \begin{itemize}
      \item Integración de \texttt{leaflet.heat}
      \item Carga dinámica del plugin para optimizar rendimiento
    \end{itemize}
    
    \item \textbf{Procesamiento de datos}
    \begin{itemize}
      \item Agrupación espacial en grid de 500m x 500m
      \item Normalización temporal (delitos por mes, no acumulación total)
      \item Cálculo de factor de concentración (bonus si hay delitos en pocos días)
      \item Normalización por percentiles (P10, P25, P50, P75, P90, P95, P99)
    \end{itemize}
    
    \item \textbf{Optimizaciones}
    \begin{itemize}
      \item Muestreo cuando hay >10k puntos
      \item Controles de intensidad ajustable (10-100\%)
      \item Modo "solo zonas críticas"
    \end{itemize}
  \end{enumerate}
\end{frame}

\begin{frame}{Fase 9: Zonas de Seguridad}
  \begin{enumerate}
    \item \textbf{Grid de celdas}
    \begin{itemize}
      \item División de CDMX en celdas de 1km²
      \item Verificación de qué celdas están dentro de los límites usando point-in-polygon
      \item Generación de todas las celdas válidas (no solo las con delitos)
    \end{itemize}
    
    \item \textbf{Clasificación}
    \begin{itemize}
      \item Cálculo de densidad de delitos por km² en cada celda
      \item Sistema de 5 niveles: muy seguro (verde), seguro, moderado, peligroso, muy peligroso (rojo)
      \item Áreas sin delitos mostradas explícitamente en verde
    \end{itemize}
    
    \item \textbf{Renderizado}
    \begin{itemize}
      \item Rectángulos semitransparentes con colores según nivel
      \item Tooltips informativos al hacer hover
      \item Limpieza de capas al desactivar
    \end{itemize}
  \end{enumerate}
\end{frame}

\begin{frame}{Fase 10: Sistema de Filtros y Controles}
  \begin{enumerate}
    \item \textbf{Panel lateral (Sidebar)}
    \begin{itemize}
      \item Checkboxes para tipos de delitos
      \item Toggles para capas (mapa de calor, zonas de seguridad, reportes)
      \item Controles de intensidad para mapa de calor
      \item Selector de fechas
    \end{itemize}
    
    \item \textbf{Estado global}
    \begin{itemize}
      \item Gestión de filtros en componente App principal
      \item Propagación de cambios a componentes hijos
      \item Actualización reactiva del mapa según filtros
    \end{itemize}
    
    \item \textbf{Estadísticas}
    \begin{itemize}
      \item Contador total de reportes ciudadanos
      \item Contador de reportes últimos 30 días
      \item Visualización en tiempo real en el sidebar
    \end{itemize}
  \end{enumerate}
\end{frame}

\begin{frame}{Fase 11: Optimizaciones y Mejoras}
  \begin{enumerate}
    \item \textbf{Rendimiento}
    \begin{itemize}
      \item Implementación de \texttt{useMemo} para cálculos costosos
      \item Lazy loading de plugins pesados (leaflet.heat)
      \item Limpieza adecuada de capas al desactivar
      \item Muestreo inteligente de datos para mapas de calor
    \end{itemize}
    
    \item \textbf{Experiencia de usuario}
    \begin{itemize}
      \item Zoom automático al activar modo de selección en mapa
      \item Animaciones suaves en transiciones
      \item Indicadores visuales claros (cursor crosshair al seleccionar)
      \item Mensajes informativos durante selección
    \end{itemize}
  \end{enumerate}
\end{frame}

\begin{frame}{Fase 12: Resolución de Problemas}
  \begin{enumerate}
    \item \textbf{Problemas técnicos resueltos}
    \begin{itemize}
      \item Tipos de TypeScript para librerías sin definiciones
      \item Iconos de Leaflet en React (solución con CDN)
      \item Normalización de datos para evitar "todo en rojo"
      \item Rendimiento con grandes volúmenes de datos
    \end{itemize}
    
    \item \textbf{Mejoras iterativas}
    \begin{itemize}
      \item Ajuste de algoritmos de normalización según feedback
      \item Refinamiento de clasificación de zonas de seguridad
      \item Optimización de grid y tamaño de celdas
    \end{itemize}
  \end{enumerate}
\end{frame}

% ============================================
% SECCIÓN 3: STACK TECNOLÓGICO
% ============================================
\section{Stack Tecnológico}

\begin{frame}{Tecnologías Utilizadas}
  \begin{itemize}
    \item \textbf{Frontend}: React 18.2, TypeScript 5.2, Vite 5.0
    \item \textbf{Estilos}: Tailwind CSS 3.3
    \item \textbf{Mapas}: Leaflet.js 1.9, react-leaflet 4.2, leaflet.heat 0.2
    \item \textbf{Análisis geoespacial}: Turf.js 6.5
    \item \textbf{Procesamiento de datos}: Python (pandas, geojson)
    \item \textbf{Almacenamiento}: LocalStorage (frontend)
  \end{itemize}
\end{frame}

\begin{frame}{Arquitectura del Código}
  \textbf{Estructura modular:}
  \begin{itemize}
    \item \texttt{components/Map/} - Componentes del mapa (MapView, HeatmapLayer, SafetyZonesLayer)
    \item \texttt{components/ReportForm/} - Formulario de reportes
    \item \texttt{components/Sidebar/} - Panel de filtros
    \item \texttt{hooks/} - Lógica reutilizable (useDelitosData)
    \item \texttt{utils/} - Utilidades (reportes.ts para LocalStorage)
    \item \texttt{types/} - Definiciones TypeScript
  \end{itemize}
\end{frame}

% ============================================
% SECCIÓN 4: RESULTADOS Y FUNCIONALIDADES
% ============================================
\section{Resultados}

\begin{frame}{Funcionalidades Implementadas}
  \begin{itemize}
    \item Mapa interactivo con límites exactos de CDMX
    \item Mapa de calor con normalización temporal y por percentiles
    \item Zonas de seguridad con clasificación en 5 niveles
    \item Sistema de reportes ciudadanos con almacenamiento local
    \item Filtros avanzados por tipo, fecha y alcaldía
    \item Sistema de marcadores personalizados por tipo de delito
    \item Zoom automático en modo de selección de ubicación
  \end{itemize}
\end{frame}

\begin{frame}{Algoritmos Clave}
  \textbf{Normalización del mapa de calor:}
  \begin{enumerate}
    \item Agrupación espacial (grid 500m x 500m)
    \item Cálculo de tasa por mes (delitos/meses totales)
    \item Factor de concentración (bonus si concentrados en pocos días)
    \item Normalización por percentiles estadísticos
    \item 10 niveles de intensidad en gradiente de colores
  \end{enumerate}
  
  \vspace{0.3cm}
  
  \textbf{Clasificación de zonas:}
  \begin{itemize}
    \item Grid de 1km² cubriendo toda CDMX
    \item Cálculo de densidad (delitos/km²)
    \item 5 niveles de clasificación con umbrales definidos
  \end{itemize}
\end{frame}

% ============================================
% SECCIÓN 5: CONCLUSIONES
% ============================================
\section{Conclusiones}

\begin{frame}{Logros del Proyecto}
  \begin{itemize}
    \item Procesamiento exitoso de 808k+ registros a formato web usable
    \item Implementación de 3 capacidades GIS requeridas (captura, consulta, análisis)
    \item Visualización eficiente de grandes volúmenes de datos geoespaciales
    \item Sistema modular y mantenible con TypeScript
    \item Optimizaciones de rendimiento efectivas
  \end{itemize}
\end{frame}

\begin{frame}{Aprendizajes y Desafíos}
  \begin{itemize}
    \item Importancia de normalización de datos para visualización efectiva
    \item Optimización de rendimiento con datasets grandes
    \item Operaciones geoespaciales complejas (unión de polígonos, point-in-polygon)
    \item Integración de múltiples librerías geoespaciales
    \item Iteración basada en feedback para mejorar algoritmos
  \end{itemize}
\end{frame}

\begin{frame}{Mejoras Futuras}
  \begin{itemize}
    \item Backend para reportes (actualmente LocalStorage)
    \item Buffers de riesgo alrededor de delitos graves
    \item Búsqueda por dirección o colonia
    \item Machine Learning para predicción de riesgo
    \item Sistema de alertas por proximidad
  \end{itemize}
\end{frame}

\begin{frame}{}
  \begin{center}
    \Large Preguntas
    
    \vspace{1cm}
    
    \normalsize Gracias por su atención
  \end{center}
\end{frame}

\end{document}
